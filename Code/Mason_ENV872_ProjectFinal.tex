\PassOptionsToPackage{unicode=true}{hyperref} % options for packages loaded elsewhere
\PassOptionsToPackage{hyphens}{url}
%
\documentclass[12pt,]{article}
\usepackage{lmodern}
\usepackage{amssymb,amsmath}
\usepackage{ifxetex,ifluatex}
\usepackage{fixltx2e} % provides \textsubscript
\ifnum 0\ifxetex 1\fi\ifluatex 1\fi=0 % if pdftex
  \usepackage[T1]{fontenc}
  \usepackage[utf8]{inputenc}
  \usepackage{textcomp} % provides euro and other symbols
\else % if luatex or xelatex
  \usepackage{unicode-math}
  \defaultfontfeatures{Ligatures=TeX,Scale=MatchLowercase}
    \setmainfont[]{Times New Roman}
\fi
% use upquote if available, for straight quotes in verbatim environments
\IfFileExists{upquote.sty}{\usepackage{upquote}}{}
% use microtype if available
\IfFileExists{microtype.sty}{%
\usepackage[]{microtype}
\UseMicrotypeSet[protrusion]{basicmath} % disable protrusion for tt fonts
}{}
\IfFileExists{parskip.sty}{%
\usepackage{parskip}
}{% else
\setlength{\parindent}{0pt}
\setlength{\parskip}{6pt plus 2pt minus 1pt}
}
\usepackage{hyperref}
\hypersetup{
            pdftitle={Changes on the Elwha River During Dam Removal},
            pdfauthor={Kathleen Mason},
            pdfborder={0 0 0},
            breaklinks=true}
\urlstyle{same}  % don't use monospace font for urls
\usepackage[margin=2.54cm]{geometry}
\usepackage{longtable,booktabs}
% Fix footnotes in tables (requires footnote package)
\IfFileExists{footnote.sty}{\usepackage{footnote}\makesavenoteenv{longtable}}{}
\usepackage{graphicx,grffile}
\makeatletter
\def\maxwidth{\ifdim\Gin@nat@width>\linewidth\linewidth\else\Gin@nat@width\fi}
\def\maxheight{\ifdim\Gin@nat@height>\textheight\textheight\else\Gin@nat@height\fi}
\makeatother
% Scale images if necessary, so that they will not overflow the page
% margins by default, and it is still possible to overwrite the defaults
% using explicit options in \includegraphics[width, height, ...]{}
\setkeys{Gin}{width=\maxwidth,height=\maxheight,keepaspectratio}
\setlength{\emergencystretch}{3em}  % prevent overfull lines
\providecommand{\tightlist}{%
  \setlength{\itemsep}{0pt}\setlength{\parskip}{0pt}}
\setcounter{secnumdepth}{5}
% Redefines (sub)paragraphs to behave more like sections
\ifx\paragraph\undefined\else
\let\oldparagraph\paragraph
\renewcommand{\paragraph}[1]{\oldparagraph{#1}\mbox{}}
\fi
\ifx\subparagraph\undefined\else
\let\oldsubparagraph\subparagraph
\renewcommand{\subparagraph}[1]{\oldsubparagraph{#1}\mbox{}}
\fi

% set default figure placement to htbp
\makeatletter
\def\fps@figure{htbp}
\makeatother

\usepackage{etoolbox}
\makeatletter
\providecommand{\subtitle}[1]{% add subtitle to \maketitle
  \apptocmd{\@title}{\par {\large #1 \par}}{}{}
}
\makeatother

\title{Changes on the Elwha River During Dam Removal}
\providecommand{\subtitle}[1]{}
\subtitle{\url{https://github.com/kfm20/DataProject_ElwhaRiver.git}}
\author{Kathleen Mason}
\date{}

\begin{document}
\maketitle

\newpage
\tableofcontents 
\newpage
\listoffigures 
\newpage

\hypertarget{rationale-and-research-questions}{%
\section{Rationale and Research
Questions}\label{rationale-and-research-questions}}

Dams were popping up across America in the early 1900's as ways to
control river flows for water usage, and power. However, in the late
1900's and early 2000's dams no longer serve an important role, and
caused more harm than good. Whether limiting sediment deposition
downstream, increasing algal blooms or evaporation in the reservoirs, or
restricting upstream access for fish species, dams are more frequently
being removed. Removal of dams can change the form and function of a
river or stream. The Elwha River flowing through Washington state,
underwent a two part dam removal process. The Glines Canyon and Elwha
Dam were removed by 2014 after a 3 year project. With the Glines Canyon
Dam in place for 90 years, 30 million tons of sediment input from the
upstream Olympic Mountians had accumulated behind the dam (USGS 2018).
With more frequent dam removal projects, it is increasingly important to
understand the potential effects dam removal can have on a river
ecosysem. Whether these effects stem from the buildup of sediment or
water behind a dam, or a newly forming floodplain along the river or at
the delta from these flow and discharge changes. While every river will
respond differently to dam removal, understanding the effects from the
project on the Elwha River can add to a greater understanding and
analysis of stream response to dam removal.

Part of the mission of the U.S. Geological Survey is focused on water
resources. They are dedicated to researching and documenting streamflow,
and water use and availability. As a credible research agency, the daily
sediment loads dataset was collected for the purpose of understanding
the changes within flow and sediment occuring during the dam removal
process along the Elwha. The dam was collected at a gage station
downstream of both dams, which is useful to see the downstream impact of
the water and sediment that had been captured behind the dams. The
dataset is part of a larger USGS study that created a 5 year sediment
budget and morphodynamic analysis of the Elwha River following the two
dam removals. My analysis uses the USGS sediment load dataset to
understand the potential changes in water flow and sediment occuring in
the Elwha River during and after the two part dam removal process from
2011 to 2016.

\textbf{Research Questions:}

*How does water and sediment discharge in the Elwha River differ during
and after the two part dam removal process?

*Can sediment discharge be predicted from water flow on the Elwha River?

*Does daily water discharge predict total sediment discharge during and
after the two part dam removal process?

\newpage

\hypertarget{dataset-information}{%
\section{Dataset Information}\label{dataset-information}}

Data was collected daily at the USGS gaging station with identification
number 12046260, located on the Elwha River at the diversion near Port
Angeles, Washington. This gage site is located downstream of both the
Elwha and Glines Canyon Dam. Data was published February 7, 2020 by USGS
and accessed April 20, 2020. The data was collected from September 15,
2011 to September 30, 2016. The dataset contains daily measurements and
estimates of discharge and various parameters of sediment loads.
Sediment load parameters include suspended concentration, loads of
suspended fine-grained particles and sand particles, and gauged bedload
for particles from 0mm to 16mm. The dataset is part of a larger study
that created a 5 year sediment budget and morphodynamic analysis of the
Elwha River following the two dam removals. Research into the time frame
of the two part dam removal process supplied dates for time stamps and
classification of ``during'' the dam removal process, and ``after''
completion of the project. Removal of the Glines Canyon Dam began on
September 15, 2011, and September 17, 2011 for the Elwha Dam (NPS 2019).
The entire project was completed on August 26, 2014 (National Park Trips
2015).Therefore, ``during'' is classified as September 15, 2011 to
August 26, 2014, and ``after'' is classified as August 27, 104 until the
end of sampling, September 30, 2016.

The dataset, coming from USGS, was already pretty neat, no NAs or
unneeded symbols were found. The dataset was simply wrangled for a best
fit in order to perform useful analyses answering the research questions
at hand. To wrangle the data, the most important variables for an
analysis focused on observing general changes on the Elwha River during
and after the dam removal project were selected. The date column was
specified as a date so R could properly understand its role in the
dataset. Columns were recoded with shorter and more coding friendly
names. The names were made more succinct and without the units in the
name. The next big wrangling task included dividing the data into two
new datasets by date, following the ``during'' and ``after'' time stamps
mentioned. A new column was created with a character class of text, with
the appropriate ``during,'' or ``after'' notation signified by the date
time frame. These two individual datasets were saved as new processed
files to be used in an analysis question later. They were also combined
again, giving a dataset of all the sampling points across the entire 5
year sampling period, just with the new time stamp column.

\newpage

\begin{longtable}[]{@{}lll@{}}
\toprule
\begin{minipage}[b]{0.30\columnwidth}\raggedright
Column name\strut
\end{minipage} & \begin{minipage}[b]{0.46\columnwidth}\raggedright
Data Description\strut
\end{minipage} & \begin{minipage}[b]{0.16\columnwidth}\raggedright
Associated Units\strut
\end{minipage}\tabularnewline
\midrule
\endhead
\begin{minipage}[t]{0.30\columnwidth}\raggedright
Date\strut
\end{minipage} & \begin{minipage}[t]{0.46\columnwidth}\raggedright
Date of daily recording\strut
\end{minipage} & \begin{minipage}[t]{0.16\columnwidth}\raggedright
YYYY-MM-DD\strut
\end{minipage}\tabularnewline
\begin{minipage}[t]{0.30\columnwidth}\raggedright
DailyDischarge\strut
\end{minipage} & \begin{minipage}[t]{0.46\columnwidth}\raggedright
Daily water discharge from the river\strut
\end{minipage} & \begin{minipage}[t]{0.16\columnwidth}\raggedright
m3/s\strut
\end{minipage}\tabularnewline
\begin{minipage}[t]{0.30\columnwidth}\raggedright
DailySSC\strut
\end{minipage} & \begin{minipage}[t]{0.46\columnwidth}\raggedright
Daily suspended sediment concentration\strut
\end{minipage} & \begin{minipage}[t]{0.16\columnwidth}\raggedright
mg/L\strut
\end{minipage}\tabularnewline
\begin{minipage}[t]{0.30\columnwidth}\raggedright
DailySuspendedSediment\strut
\end{minipage} & \begin{minipage}[t]{0.46\columnwidth}\raggedright
Daily total suspended sediment load\strut
\end{minipage} & \begin{minipage}[t]{0.16\columnwidth}\raggedright
tonnes\strut
\end{minipage}\tabularnewline
\begin{minipage}[t]{0.30\columnwidth}\raggedright
DailySSfines\strut
\end{minipage} & \begin{minipage}[t]{0.46\columnwidth}\raggedright
Daily suspended fine grained sediment load\strut
\end{minipage} & \begin{minipage}[t]{0.16\columnwidth}\raggedright
tonnes\strut
\end{minipage}\tabularnewline
\begin{minipage}[t]{0.30\columnwidth}\raggedright
DailySSsand\strut
\end{minipage} & \begin{minipage}[t]{0.46\columnwidth}\raggedright
Daily suspended sand sediment load\strut
\end{minipage} & \begin{minipage}[t]{0.16\columnwidth}\raggedright
tonnes\strut
\end{minipage}\tabularnewline
\begin{minipage}[t]{0.30\columnwidth}\raggedright
TotalSedimentDischarge\strut
\end{minipage} & \begin{minipage}[t]{0.46\columnwidth}\raggedright
Total daily sediment discharge\strut
\end{minipage} & \begin{minipage}[t]{0.16\columnwidth}\raggedright
tonnes\strut
\end{minipage}\tabularnewline
\begin{minipage}[t]{0.30\columnwidth}\raggedright
Projectyear\strut
\end{minipage} & \begin{minipage}[t]{0.46\columnwidth}\raggedright
Year of sampling project, extending from 1-5 years\strut
\end{minipage} & \begin{minipage}[t]{0.16\columnwidth}\raggedright
year\strut
\end{minipage}\tabularnewline
\begin{minipage}[t]{0.30\columnwidth}\raggedright
TimeStamp\strut
\end{minipage} & \begin{minipage}[t]{0.46\columnwidth}\raggedright
Distinguishes time frame in the dam removal process\strut
\end{minipage} & \begin{minipage}[t]{0.16\columnwidth}\raggedright
``during'' or ``after''\strut
\end{minipage}\tabularnewline
\bottomrule
\end{longtable}

\newpage

\hypertarget{exploratory-analysis}{%
\section{Exploratory Analysis}\label{exploratory-analysis}}

An initial exploratory analysis is conducted to see general trends in
data related to water and sediment discharge, and suspended
concentrations in Elwha River during and after the dam removal process.
Daily water discharge from the river, \emph{Figure 1}, appears to have
higher peaks of discharge in 2015 and 2016, the years after the dam
removal project was complete. Embedding the dates involved with the dam
removal process, such as start of removal, and completion of each dam
removal will help differentiate differences in discharge related to more
open flows with dams removed. This data might need to be looked at a
different scale, instead of daily, maybe monthly averages will show a
different relationship, or a similar one at a different magnitude.

\begin{figure}
\centering
\includegraphics{Mason_ENV872_ProjectFinal_files/figure-latex/Exploratory Analysis Figure 1-1.pdf}
\caption{Daily water discahrge of the Elwha River, WA, from September
15, 2011 to September 30, 2016.}
\end{figure}

\begin{figure}
\centering
\includegraphics{Mason_ENV872_ProjectFinal_files/figure-latex/Exploratory Analysis Figure 2-1.pdf}
\caption{Daily suspended sediment in the Elwha River, WA, from September
15, 2011 to September 30, 2016.}
\end{figure}

\begin{figure}
\centering
\includegraphics{Mason_ENV872_ProjectFinal_files/figure-latex/Exploratory Analysis Figure 3-1.pdf}
\caption{Daily suspended sediment of fine-grained particles in the Elwha
River, WA, from September 15, 2011 to September 30, 2016.}
\end{figure}

\begin{figure}
\centering
\includegraphics{Mason_ENV872_ProjectFinal_files/figure-latex/Exploratory Analysis Figure 4-1.pdf}
\caption{Daily suspended sediment of sand particles in the Elwha River,
WA, from September 15, 2011 to September 30, 2016.}
\end{figure}

\begin{figure}
\centering
\includegraphics{Mason_ENV872_ProjectFinal_files/figure-latex/Exploratory Analysis Figure 5-1.pdf}
\caption{Daily suspended sediment of fine-grained particles (blue) and
sand particles (gray) in the Elwha River, WA, from September 15, 2011 to
September 30, 2016.}
\end{figure}

\newpage

Suspended sediment concentrations might give a sense of the velocity of
the flow heading downstream, and how much sand was stuck behind the dams
that is then in movement after their removal. Looking at suspended
concentrations over time may show how long it takes for the sediment
behind the dam to resettle in the river, allowing the river to reach a
new morphological norm. General trends of suspended sediments,
\emph{Figure 2}, show more tonnes happening around the year 2013, which
is during the removal of the Elwha Dam, and the Glines Canyon Dam had
already been removed. However, there still exists some high recordings
of suspended sediment later on through the project years. However, these
might have to do with the high water discharge. Further analysis will
compare the relationship between daily suspended sediment and water
discharge over time. The dataset also has daily suspended sediment of
fine-grained particles, and sand particles. Their general point plots
are shown individually, \emph{Figures 3 and 4} and together in one plot,
\emph{Figure 5}, where we see there doesn't appear to be much difference
between the makeup of the suspended sediment, although a further test
can prove this.

\begin{figure}
\centering
\includegraphics{Mason_ENV872_ProjectFinal_files/figure-latex/Exploratory Analysis Figure 6-1.pdf}
\caption{Daily total sediment discharge from the Elwha River, WA, from
September 15, 2011 to September 30, 2016.}
\end{figure}

\begin{figure}
\centering
\includegraphics{Mason_ENV872_ProjectFinal_files/figure-latex/Exploratory Analysis Figure 7-1.pdf}
\caption{Daily total sediment discharge and water discharge on the Elwha
River, with a linear model, from September 15, 2011 to September 30,
2016.}
\end{figure}

\newpage

While there are multiple parameters highlighting the sediment traveling
downstream, daily sediment discharge is a straight forward parameter of
moving sediment in the Elwha River during and after the dam removal
processes. From a general line plot of total sediment discharge over
time, \emph{Figure 6}, shows a peak discharge in 2013 and multiple large
peaks as well as what appears like a larger avergae sediment discharge
happening after 2014. These seem to make sense with the time stamps of
the dam removal process, but combining the time stamps and this
relationship in one graph will help better visualize the relationship
with dam removal over time. Calculating the mean discharges during
removal and after would be useful to show changes in discharge across
the dam removal process.

With daily water discharge and total sediment discharge being important
parameters for showing changes in the Elwha River following dam removal,
the relationship between them was graphed with a linear model to show a
relationship, \emph{Figure 7}. It makes sense that increased flow would
generate increased sediment discharge, which we see from the positive
linear relationship. It would be interesting to see this relationship
graphed out for during and after dam removal and see how this
relationship might change.

\newpage

\hypertarget{analysis}{%
\section{Analysis}\label{analysis}}

\hypertarget{question-1-how-does-water-and-sediment-discharge-in-the-elwha-river-differ-during-and-after-the-two-part-dam-removal-process}{%
\subsection{Question 1: How does water and sediment discharge in the
Elwha River differ during and after the two part dam removal
process?}\label{question-1-how-does-water-and-sediment-discharge-in-the-elwha-river-differ-during-and-after-the-two-part-dam-removal-process}}

A closer look into total sediment discharge and daily discharge of water
from the Elwha River with attention on the time stamps of when the dam
removal project begigs, and when it is completed, \emph{Figures 8 and
9}, prompted an in depth analysis of trends. Data was separated into
during the dam removal process and after its completion, September 26,
2014. Part one of this analysis hoped to determine whether the means of
daily water discharge were equivalent during and after the dam removal
process. similarly, part two hoped to determine whether the means of
daily total sediment discharge were equivalent during and after the dam
removal process.

\begin{figure}
\centering
\includegraphics{Mason_ENV872_ProjectFinal_files/figure-latex/Intro to Question (Figure 8)-1.pdf}
\caption{Daily water discharge from the Elwha River from September 15,
2011 to September 30, 2016 measured at the U.S. Geological Survey gaging
station 12046260 at the diversion near Port Angeles, Washington. A
project to remove the Elwha and Glines Canyon Dam began on September 15,
2011, and was completed on August 26, 2014. Mean Daily discharge across
the whole time range was 47.7.}
\end{figure}

\newpage

\begin{figure}
\centering
\includegraphics{Mason_ENV872_ProjectFinal_files/figure-latex/Intro to Question (Figure 9)-1.pdf}
\caption{Daily total sediment discharge from the Elwha River from
September 15, 2011 to September 30, 2016 measured at the U.S. Geological
Survey gaging station 12046260 at the diversion near Port Angeles,
Washington. A project to remove the Elwha and Glines Canyon Dam began on
September 15, 2011, and was completed on August 26, 2014. Mean Daily
sediment discharge across the whole time range was 9886.876 tonnes.}
\end{figure}

\newpage

Daily water discharge data, \emph{Figure 8}, was classified as during or
after the dam removal process based on dates. September 26, 2014 was the
dividing date. A two-sample t-test was run to determine if means are
equivalent during and after dam removal. This test assumes equal
variance, however, results showed there is not an equal variance,
meaning the assumption of normality is not met (p-value \textless{}
0.05; Shapiro-Wilk normality test). Similarly, Daily total sediment
discharge data, \emph{Figure 9}, was classified as during or after the
dam removal process based on dates, and a two-sample t-test was also
run. Daily total sediment discharge over time does not have equal
variance as well (p-value \textless{}0.05; Shapiro-Wilk normality test).

To avoid the assumption or normality, a non-parametric method, Wilcoxon
rank sum, is used to determine if means are equivalent during and after
dam removal for both water and sediment discharge. The mean daily
discharge during the dam removal process is 48.31 m2/s, and 46.9 m2/s
after the completion of the project. There is a significant difference
between the means of daily water discharge during and after dam removal
(\emph{Figure 10}, W = 349134, p-value \textless{} 0.0001, Wilcoxon rank
sum test). The mean daily total sediment discharge during the dam
removal process is 11,319.0 tonnes, and 7,888.0 tonnes for after
completion of dam removal. There is a significant difference between the
means of daily total sediment discharge during and after dam removal
(\emph{Figure 11}, W = 209858, p-value \textless{}0.0001, Wilcoxon rank
sum test).

\newpage

\begin{figure}
\centering
\includegraphics{Mason_ENV872_ProjectFinal_files/figure-latex/Two Way T-Test REsults Water (Figure 10)-1.pdf}
\caption{Daily water discharge distribution during and after the Elwha
River two dam removal process. During the dam removal is classified by
dates from September 15, 2011 to August 26, 2014, and after is from then
until September 30, 2016.}
\end{figure}

\begin{figure}
\centering
\includegraphics{Mason_ENV872_ProjectFinal_files/figure-latex/Two Way T-Test REsults sediment(Figure 11)-1.pdf}
\caption{Daily total sediment discharge distribution during and after
the Elwha River two dam removal process. During the dam removal is
classified by dates from September 15, 2011 to August 26, 2014, and
after is from then until September 30, 2016.}
\end{figure}

\newpage

\hypertarget{question-2-can-sediment-discharge-be-predicted-from-water-flow-on-the-elwha-river}{%
\subsection{Question 2: Can sediment discharge be predicted from water
flow on the Elwha
River?}\label{question-2-can-sediment-discharge-be-predicted-from-water-flow-on-the-elwha-river}}

Increased water flow on a river should carry more sediment, producing
more overall sediment discharge. An analysis of water discharge and
sediment discharge is performed over the entirity of the sampling period
to find a general trend of the relationship of these two parameters over
time on the Elwha.

\begin{figure}
\centering
\includegraphics{Mason_ENV872_ProjectFinal_files/figure-latex/Linear Regression (figure 12)-1.pdf}
\caption{Daily water discharge as an indicator for daily total sediment
discharge on the Elwha River, with a linear regression.}
\end{figure}

A linear regression model determines a line of best fit between two
continuous variables, in this case, daily water discharge and total
sediment discharge. There is a strong, significant positive correlation
between water and sediment discharge during the entirity of the sampling
period (cor= 0.73, p-value \textless{} 0.0001, pearson's correlation
test). Therefore, water discharge is an effedctive predictor for
sediment discharge (\emph{Figure 12}, p-value \textless{} 0.0001, R2=
0.54, linear regression). Each square meter per second of water
discharge accounted for 0.0008 tonnes of sediment discharge.

\newpage

\hypertarget{question-3-does-daily-water-discharge-predict-total-sediment-discharge-during-and-after-the-two-part-dam-removal-process}{%
\subsection{Question 3: Does daily water discharge predict total
sediment discharge during and after the two part dam removal
process?}\label{question-3-does-daily-water-discharge-predict-total-sediment-discharge-during-and-after-the-two-part-dam-removal-process}}

Data recorded during the dam removal process, and data recorded after
were separated. Separate linear regressions were run on each to
determine if water discharge is still an effective predictor of sediment
discharge during these separate periods, and to what extent.

\begin{figure}
\centering
\includegraphics{Mason_ENV872_ProjectFinal_files/figure-latex/Linear During (Figure 13)-1.pdf}
\caption{Daily water discharge as an indicator for daily total sediment
discharge on the Elwha River during the two part dam removal process,
with a linear regression.}
\end{figure}

\begin{figure}
\centering
\includegraphics{Mason_ENV872_ProjectFinal_files/figure-latex/Linear After (Figure 14)-1.pdf}
\caption{Daily water discharge as an indicator for daily total sediment
discharge on the Elwha River after completion of the two part dam
removal process, with a linear regression.}
\end{figure}

\newpage

There is a strong, significant positive correlation between water and
sediment discharge during the dam removal process (cor= 0.70, p-value
\textless{} 0.0001, pearson's correlation test). During dam removal,
water discharge is an effective predictor for sediment discharge
(\emph{Figure 13}, p-value \textless{} 0.0001, R2= 0.50, linear
regression). Each square meter per second of water discharge accounted
for 0.0007 tonnes of sediment discharge.

After the completion of dam removal, there is still a strong,
significant positive correlation between water and sediment discharge
(cor= 0.77, p-value \textless{} 0.0001, pearson's correlation test).
Water flow serves as an effective predictor for sediment discharge
(\emph{Figure 14}, p-value \textless{} 0.0001, R2= 0.59, linear
regression). Each square meter per second of water discharge accounted
for 0.001 tonnes of sediment discharge.

There is a stronger correlation after dam removal completion, making
water discharge a better predictor of sediment discharge after dam
removal compared to during the process. The model accounts for a larger
percentage of variance by the explanatory variable of water discharge
after dam removal completion.

\newpage

\hypertarget{summary-and-conclusions}{%
\section{Summary and Conclusions}\label{summary-and-conclusions}}

There were significant differences in parameters measured during dam
removal and after the projects completion. While daily water discharge
appeared similar in their reported means, there was still a significant
difference between during and after dam removal. After completion, there
appears to be more high peaks of flow, \emph{Figures 8 and 10}. This
makes sense because there is no longer two dams in place that would have
controled flows to have less of high peak flow events. Now without a dam
to create a reservoir stock of water, the Elwha River might experience
more high peak flow events, especially as the form of the river and
floodplain re-adjust to recieving new water and sediment input.

Water discharge is an effective predictor of sediment discharge. Through
analysis this was proven to be true throughout the entirity of the
sampling time frame, as well as during the dam removal process and after
it's completion. After completion of the project, each unit of water
flow accounted for higher amounts of sediment discharge compared to
during. Again, this may be due to the higher peak flows carrying more
sediment downstream, as well as the higher sediment discharge that
occured after the dams were fully removed.

Overall, it makes sense that sediment discharge and water discharge
reach higher peaks without the dams in place. Not only are the dams not
in place to hold back water and sediment, but now after it's removal,
everything that was behind the dam is released. While this analysis
provides an understanding of changes in discharge during and a couple
years after completion with a fully flwoing river, more years of data
collection can be useful. Two years of data does not seem to be enough
time to see graphically a new stability of water flow and sediment
discharge. More years of data may be useful to see the Elwha River reach
a new mean daily norm of these discharge parameters following dam
removal.

Lastly, a lot can be learned from the steady increase of both peak
sediment and water discharge in the early stages of the removal process.
Specific attention should be given to total daily sediment discharge
over time, \emph{Figure 9}, and the drastic changes in peaks since the
start of dam removal. With 30 million tons of sediment recorded to be
held behind the dams, from this figure the release of it is graphically
displayed. Additionally, increased flows following dam removal might
cause more erosion from the floodplain it may be newly interacting with.
This would add more sediment to the total discharge, on top of what was
already stored behind the dam. Sediment load is important because of the
ecological impacts on species, and the formation of floodplains and the
delta the Elwha River drains into.

\newpage

\hypertarget{references}{%
\section{References}\label{references}}

Ritchie, A.C., Curran, C.A., Magirl, C.S., Bountry, J.A., Hilldale,
R.C., Randle, T.J., and Duda, J.J., 2018, Data in support of 5-year
sediment budget and morphodynamic analysis of Elwha River following dam
removals: U.S. Geological Survey data release,
\url{https://doi.org/10.5066/F7PG1QWC}.

National Park Service. 2019. Elwha River Restoration: Dam Removal.
Olympic National Park Washington, National Park Service,
\url{https://nps.gov/olym}

National Park Trips. 2015. Saving salmon by demolishing dams on the
Elwha River. Olympic National Park Trips,
\url{https://myolympicpark.com}

United States Geologic Survey. 2018. Moving Mountains: Elwha River Still
Changing Five Years After World's Largest Dam-Removal Project: More than
20 million tons of sediment flushed to the sea. Department of the
Interior, US Geological Survey, Reston, VA. \url{https://www.usgs.gov}

\end{document}
